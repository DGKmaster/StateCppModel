\documentclass[14pt,a4paper]{extreport}

\usepackage[left=30mm, top=20mm, right=10mm, bottom=20mm]{geometry}
\usepackage{cmap}		
\usepackage[utf8]{inputenc}
\usepackage[english, russian]{babel}
\usepackage{framed}
\usepackage{amsmath}
\usepackage{graphicx}
\usepackage{wrapfig}
\usepackage{listings}
\usepackage{color}
\usepackage{indentfirst}
\usepackage{textcomp}
\usepackage{titlesec}
\usepackage{alltt}

\usepackage{cite}
\usepackage{url}
%%%%%%%%%%%%%%%%%%%%%%%%%%%%%%%%%%%%%%%%%%%%%%%%%%%%%%%%%%%%%%%%%%%%%%%%

\titleformat{\chapter}[display]
    {\filcenter\large\bfseries}
    {\MakeUppercase{\chaptertitlename} \thechapter}
    {8pt}
    {\bfseries}{}
\titleformat{\section}
    {\normalsize\bfseries}
    {\thesection}
    {1em}{}
\titleformat{\subsection}
    {\normalsize\bfseries}
    {\thesubsection}
    {1em}{}

%% Настройка вертикальных и горизонтальных отступов
\titlespacing*{\chapter}{0pt}{-30pt}{8pt}
\titlespacing*{\section}{\parindent}{*4}{*4}
\titlespacing*{\subsection}{\parindent}{*4}{*4}

%% Отступ от левого края
\oddsidemargin=0pt 

\lstdefinestyle{customc}{belowcaptionskip=1\baselineskip,breaklines=true,frame=L,xleftmargin=\parindent,  language=C, showstringspaces=false, basicstyle=\footnotesize\ttfamily,keywordstyle=\bfseries\color{green!40!black},  commentstyle=\itshape\color{purple!40!black}, identifierstyle=\color{blue}, stringstyle=\color{orange},numbers=left,numbersep=12pt, numberstyle=\small\color{mygray},}
\lstset{escapechar=@,style=customc}

\newcommand{\HRule}{\rule{\linewidth}{0.5mm}}

\begin{document}

%% Титульный
%%%%%%%%%%%%%%%%%%%%%%%%%%%%%%%%%%%%%%%%%%%%%%%%%%%%%%%%%%%%%%%%%%%%%%%%
\begin{titlepage}
\begin{center}


{\normalsize 
Федеральное государственное автономное образовательное учреждение высшего образования 
\\Санкт-Петербургский национальный исследовательский университет 
\\информационных технологий, механики и оптики}

\vspace{1.5cm}

\vspace{0.5cm}
\large Факультет систем управления и робототехники
% Upper part of the page. The '~' is needed because \\
% only works if a paragraph has started.
%\includegraphics[width=0.18\textwidth]{img/logo.png}~\\[1cm]
\vspace{2cm}

\large Курсовой проект
\\ Вариант №2

по дисциплине <<Программирование систем управления>>

{ \large \bfseries <<Моделирование системы управления>>\\[0.4cm] }

% Author and supervisor
\noindent

\begin{flushright} \normalsize
\emph{Выполнили:}\\
Студент группы R41332\\
Волков \textsc{А.~А.}\\
% Щербаков \textsc{П.~В.}\\

\emph{Проверил:} \\
Томашевич \textsc{С.~И.}
\end{flushright}

\vfill

% Bottom of the page
{\normalsize Санк-Петербург, 2019 г.}

\end{center}
\end{titlepage}

%%%%%%%%%%%%%%%%%%%%%%%%%%%%%%%%%%%%%%%%%%%%%%%%%%%%%%%%%%%%%%%%%%%%%%%%

%% Оглавление
%%%%%%%%%%%%%%%%%%%%%%%%%%%%%%%%%%%%%%%%%%%%%%%%%%%%%%%%%%%%%%%%%%%%%%%%
\newpage
\tableofcontents
%%%%%%%%%%%%%%%%%%%%%%%%%%%%%%%%%%%%%%%%%%%%%%%%%%%%%%%%%%%%%%%%%%%%%%%%

%% Задание
%%%%%%%%%%%%%%%%%%%%%%%%%%%%%%%%%%%%%%%%%%%%%%%%%%%%%%%%%%%%%%%%%%%%%%%%
\chapter*{Задание}
\addcontentsline{toc}{chapter}{Задание}

\begin{enumerate}
\item 
Реализовать класс интегратора в .cpp и .h файлах.
    
\item 
Привести задающее воздействие в виде модели с использованием интегратором.
    
\item 
Дисретизировать полученные модели задающего воздействия 
и объекта управления с шагами дискретизации 5, 50, 100 Гц.

\item
Программно реализовать отдельными классами четыре случая объектов 
(непрерывный и три дискретных). Для дискретных случаев сделать 
реализацию с использованием разностных уравнений.
\begin{equation} 
    x_{k+1} = A \cdot x_k
\end{equation}
То есть интегратор заменяется на элемент памяти.

\item 
Добавить реализованные классы в предоставленную программу для \\ QtCreator.

\item
Поочередно провести сравнение поведений реализованных непрерывв-
ных моделей с дискретными моделями с соответствующими шагами
дискретизации. Шаг дискретизации меняется в предоставленной про-
грамме. В результате должно получиться три пары сравнений.

\item 
В предоставленной программе QtCreator настроить последовательный
порт (qSerialPort) на скорость 115200 бод, формат 8N1.

\item 
Закодировать с помощью метода COBS значения, полученные с выхода
объекта в следующем формате:
{[0x0A 0xXX 0xXX 0xXX 0xXX 0xCR]}, где 0xXX – байты полученного 
числа с плавающей точкой (float) в
обратном порядке, а 0xCR – проверочная сумма, равная сумме всех
остальных байт сообщения, вычтенной из 0xFF.

\end{enumerate}
%%%%%%%%%%%%%%%%%%%%%%%%%%%%%%%%%%%%%%%%%%%%%%%%%%%%%%%%%%%%%%%%%%%%%%%%

%% Исходные данные
%%%%%%%%%%%%%%%%%%%%%%%%%%%%%%%%%%%%%%%%%%%%%%%%%%%%%%%%%%%%%%%%%%%%%%%%
\newpage
\chapter*{Исходные данные}
\addcontentsline{toc}{chapter}{Исходные данные}

\begin{equation}
    u(t) = 3 \cdot  cos(0.1 \cdot t + 1)
\end{equation}

\begin{equation}
    A = 
    \begin{bmatrix} 
        0 & 1 & 0 \\ 
        0 & 0 & 1 \\
        -1.5 & -5 & -2
    \end{bmatrix}
\end{equation}

\begin{equation}
    B = 
    \begin{bmatrix} 
        0 \\ 
        0 \\
        1
    \end{bmatrix}
\end{equation}

\begin{equation}
    C = 
    \begin{bmatrix} 
        0.5 & 0 & 0
    \end{bmatrix}
\end{equation}

\begin{equation}
\dot x = A \cdot x + B \cdot u(t)
\end{equation}

\begin{equation}
y = C \cdot x
\end{equation}
%%%%%%%%%%%%%%%%%%%%%%%%%%%%%%%%%%%%%%%%%%%%%%%%%%%%%%%%%%%%%%%%%%%%%%%%

%% Ход работы
%%%%%%%%%%%%%%%%%%%%%%%%%%%%%%%%%%%%%%%%%%%%%%%%%%%%%%%%%%%%%%%%%%%%%%%%
\chapter*{Ход работы}
\addcontentsline{toc}{chapter}{Ход работы}

Рисунок 1 - Блок-схема работы программы
%%%%%%%%%%%%%%%%%%%%%%%%%%%%%%%%%%%%%%%%%%%%%%%%%%%%%%%%%%%%%%%%%%%%%%%%

%% Приложение А. Исходный код программы
%%%%%%%%%%%%%%%%%%%%%%%%%%%%%%%%%%%%%%%%%%%%%%%%%%%%%%%%%%%%%%%%%%%%%%%%
\newpage
\chapter*{Приложение А. Исходный код программы}
\addcontentsline{toc}{chapter}{Приложение А. Исходный код программы}


\textbf{Файл model.cpp}
\begin{alltt}
\begin{verbatim}
#include "model.h"

\end{verbatim}
\end{alltt}

\textbf{Файл model.h}
\begin{alltt}
\begin{verbatim}
#include "model.h"

\end{verbatim}
\end{alltt}

\textbf{Файл main.cpp}
\begin{alltt}
\begin{verbatim}
#include "dep/qcustomplot.h"
#include "view/widget.h"
#include <cmath>

#include <QApplication>

int main(int argc, char *argv[])
{
    QApplication a(argc, argv);
    Widget w;
    w.show();

    return a.exec();
}    
\end{verbatim}
\end{alltt}


\textbf{Файл widget.h}
\begin{alltt}
\begin{verbatim}
#ifndef WIDGET_H
#define WIDGET_H

#include <QWidget>
#include <QTimer>
#include <QVector>

#ifdef __linux__
#include <sys/time.h>
#endif

#include "dep/qcustomplot.h"
#include "model/model.h"

namespace Ui {
class Widget;
}

class Widget : public QWidget
{
    Q_OBJECT

public:
    explicit Widget(QWidget *parent = nullptr);
    ~Widget();

public slots:
    void makePlot();

private:
    Ui::Widget *ui;
    QGridLayout *mainlayout;
    QCustomPlot *inputPlot;
    QCustomPlot *outputPlot;

    double startTime;
    double dt;

    QTimer *timer;
    QVector<double> time;
    QVector<double> input;
    QVector<double> output;

    // --------------------------
    // Add pointer to the object here
    // --------------------------
    Model *object;   // <=
    // --------------------------
    // Add pointer to the object here
    // --------------------------
};

#endif // WIDGET_H
\end{verbatim}
\end{alltt}

\textbf{Файл widget.cpp}
\begin{alltt}
\begin{verbatim}
#include "widget.h"
#include "ui_widget.h"
#include <iostream>
#include <cmath>

Widget::Widget(QWidget *parent) :
    QWidget(parent),
    ui(new Ui::Widget)
{
    // --------------------------
    // Create the object here
    // --------------------------
    object = new Model();   // <=
    // --------------------------
    // Create the object here
    // --------------------------

    ui->setupUi(this);

    // Set window size
    this->setFixedSize(1400,700);

    // Add main layout with two plots
    mainlayout = new QGridLayout(this);
    inputPlot = new QCustomPlot(this);
    outputPlot = new QCustomPlot(this);
    mainlayout->addWidget(inputPlot,0,0);
    mainlayout->addWidget(outputPlot,0,1);
    inputPlot->setFixedSize(this->width()/3,this->height());
    outputPlot->setFixedSize(this->width()/3,this->height());

    // Give the axes some labels:
    inputPlot->xAxis->setLabel("t");
    inputPlot->yAxis->setLabel("input");
    outputPlot->xAxis->setLabel("t");
    outputPlot->yAxis->setLabel("output");

    // --------------------------
    // Change ranges if you need
    // --------------------------
    // Set axes ranges so see all data:
    inputPlot->xAxis->setRange(0, object->SIMULATION_TIME);
    inputPlot->yAxis->setRange(-3, 3);
    outputPlot->xAxis->setRange(0, object->SIMULATION_TIME);
    outputPlot->yAxis->setRange(-3, 3);

    // Get time in msec
    // --------------------------
    // Google for MacOS timings
    // --------------------------
#ifdef __linux__
    struct timeval tmpStruct;
    gettimeofday(&tmpStruct, nullptr);
    startTime = tmpStruct.tv_sec * 1000 + tmpStruct.tv_usec / 1000 + 0.5;
#endif
#ifdef _WIN32
    SYSTEMTIME tmpStruct;
    GetSystemTime(&tmpStruct);
    startTime = tmpStruct.wSecond * 1000 + tmpStruct.wMilliseconds + 0.5;
#endif

    makePlot();
    timer = new QTimer(this);
    connect(timer, SIGNAL(timeout()), this, SLOT(makePlot()));

    // --------------------------
    // Set sampling time here
    // --------------------------
    timer->start(object->TIME_STEP);
    // --------------------------
    // Set sampling time here
    // --------------------------
}

Widget::~Widget()
{
    delete ui;
    delete inputPlot;
    delete outputPlot;
    delete timer;
    delete mainlayout;

    // --------------------------
    // Delete the object here
    // --------------------------
    delete object;
    // --------------------------
    // Delete the object here
    // --------------------------
}

void Widget::makePlot() {
// generate some data:
#ifdef __linux__
    struct timeval tmpTime;
    gettimeofday(&tmpTime, nullptr);
    double tmp = (tmpTime.tv_sec * 1000 + tmpTime.tv_usec / 1000 + 0.5)-startTime;
#endif
#ifdef _WIN32
    SYSTEMTIME tmpTime;
    GetSystemTime(&tmpTime);
    double tmp = tmpTime.wSecond * 1000 + tmpTime.wMilliseconds + 0.5 - startTime;
#endif

    // --------------------------
    // Replace input signal with ours
    // --------------------------
    // double signal = std::sin(tmp/1000);
    double signal = object->control();
    // double signal = 1;
    // object->send(signal);
    // --------------------------
    // Replace input signal with ours
    // --------------------------

    // Update input array to plot
    input.append(signal);

    // Get elapsed time
    if (time.empty()) {
        dt = 0;
    } else {
        dt = tmp / 1000.0 - time.last();
    }

    object->time_now += object->TIME_STEP;

    // Update time array to plot
    time.append(object->time_now);

    // --------------------------
    // Update the object here
    // --------------------------
    output.append(object->update(signal));
    // --------------------------
    // Update the object here
    // --------------------------

    inputPlot->addGraph();
    inputPlot->graph(0)->setData(time, input);

    outputPlot->addGraph();
    outputPlot->graph(0)->setData(time, output);

    inputPlot->replot();
    outputPlot->replot();

    if (object->time_now > object->SIMULATION_TIME) {timer->stop();}
}
\end{verbatim}
\end{alltt}
%%%%%%%%%%%%%%%%%%%%%%%%%%%%%%%%%%%%%%%%%%%%%%%%%%%%%%%%%%%%%%%%%%%%%%%%

\end{document}
